\documentclass[11pt]{article}
\usepackage{geometry}                % See geometry.pdf to learn the layout options. There are lots.
\geometry{letterpaper}                   % ... or a4paper or a5paper or ... 
%\geometry{landscape}                % Activate for for rotated page geometry
\usepackage[parfill]{parskip}    % Activate to begin paragraphs with an empty line rather than an indent
\usepackage{graphicx}
\usepackage{amssymb}
\usepackage{epstopdf}
\usepackage{amsmath}
\DeclareGraphicsRule{.tif}{png}{.png}{`convert #1 `dirname #1`/`basename #1 .tif`.png}

\title{Notes on Forced-Dissipative Flow}
\author{Jeffrey J. Early}
%\date{}                                           % Activate to display a given date or no date

\begin{document}
\maketitle


\section{Scales}

We are solving the equation
\begin{equation}
\frac{\partial }{\partial t} \left( \nabla^2 \psi - \lambda^2 \psi \right)  + J\left( \psi, \nabla^2 \psi \right) = F + \alpha \lambda^2 \psi -r \nabla^2 \psi + \nu Q(k) \nabla^4 \psi 
\end{equation}
where we have included thermal damping ($\alpha \lambda^2$), frictional damping ($r$), and hyper diffusion ($\nu$). I've included the deformation scale $\lambda^{-1}$ as part of the thermal damping coefficient to be consistent with its derivation in Scott and Dritschel (2013), even though we will not necessarily consider flows above the deformation scale.

Note---this analysis should almost certainly be done with point symmetries.

%%%%%%%%%%%%%%%%%%
%
\subsection{Forcing}
%
%%%%%%%%%%%%%%%%%%

The forcing function $F$ looks like
\begin{equation}
F(x,t) = f_\zeta(|\mathbf{k}|) \cos \left( 2\pi \mathbf{k} \cdot \mathbf{x} - \omega(|\mathbf{k}|)t \right).
\end{equation}
The forcing could be confined to a single wavenumber, $|\mathbf{k}|= k_f$, or spread out across a wider band. Similarly, the forcing could be confined to a single frequency, $\omega=\omega_f$, or, as is also commonly done, it may be spread out across a wider band, completely uncorrelated forcing from one moment to the next. We can write this function in the frequency domain by taking the Fourier transform. A single forcing wave number transforms to,
\begin{equation}
\hat{F}(k,t) = f_\zeta \frac{e^{- i \left( \phi_0+\omega t\right)}}{2} \left( \delta \left(k -k_f\right) + \delta\left(k + k_f\right)  \right)
\end{equation}
where $f_\zeta$ has units of inverse seconds squared, and the delta function has units of meters. In my usual DFT formulation, the inverse transform does not change the units. Specifically I use,
\begin{equation}
\hat{x}_n \equiv \sum_{k=0}^{N-1} x_k e^{2\pi i f_k t_n}
\end{equation}
which means that the series coefficient $f_\zeta$ carries the same units as $F(x,y)$.

If we want to make a similar function, but slightly more broadband, we should divide $f_\zeta$ by the square root of area that will be summed over in wavenumber space. In our discrete implementation, we simply need to divide by the square root of the number of wave numbers that we distributed the power over. We do this so that then $\sum F^\ast F = f_\zeta^2$.

If we want the forcing to move to a new location in a given time, then we want that $\frac{\partial \omega}{\partial k}=c$ where $c$ is the speed at which forcing function will translate. If we want the function to move a distance $k_f^{-1}$ in time $T$, then $c=\pm\left( k_f T \right)^{-1}$ and $\omega = \pm \frac{2\pi k}{k_f T }$.

%%%%%%%%%%%%%%%%%%
%
\subsection{Energy}
%
%%%%%%%%%%%%%%%%%%

The total energy in the system (let's assume a periodic domain) is seen to be,
\begin{equation}
\frac{d \hat{E}}{d t} = - \int \psi F \, dA - \alpha \lambda^2 \int \psi^2 \, dA - 2 r \hat{E} - \nu \int \left( \nabla^2 \psi \right)^2 \, dA
\end{equation}
where $2 \hat{E}= - \int \psi \nabla^2 \psi \, dA =  \int \left( \nabla \psi \right)^2 \, dA$. The rate at which energy is injected into the system must therefore be, on average, $\epsilon=\frac{1}{A} \int \psi F \, dA$. This can be found in Vallis (2006).

%%%%%%%%%%%%%%%%%%
%
\subsection{Enstrophy}
%
%%%%%%%%%%%%%%%%%%

Defining vorticity as $\zeta = \nabla^2 \psi$, the total enstrophy in the system is seen to be,
\begin{equation}
\frac{d \hat{Z}}{d t} = \int \zeta F \, dA + \alpha \lambda^2 \int \psi \zeta \, dA - 2 r \hat{Z} - \nu \int \left( \nabla \zeta \right)^2 \, dA
\end{equation}
The rate at which enstrophy is injected into the system is given by $\eta=-\frac{1}{A} \int \zeta F \, dA$. If we assume that the forcing is narrow-banded, then $\eta \approx - (2\pi k_f)^2 \epsilon$.

%%%%%%%%%%%%%%%%%%
%
\subsection{Units}
%
%%%%%%%%%%%%%%%%%%

Units of various quantities are as follows,
\begin{align}
\left[ \zeta \right] =& \frac{1}{\textrm{s}} \\
\left[ \psi \right] =& \frac{\textrm{m}^2}{\textrm{s}} \\
\left[ F \right] =& \frac{1}{\textrm{s}^2} \\
\left[ \alpha \lambda^2 \right] =& \frac{1}{\textrm{m}^2\cdot\textrm{s}} \\
\left[ r \right] =& \frac{1}{\textrm{s}} \\
\left[ \nu \right] =& \frac{\textrm{m}^2}{\textrm{s}} \\
\left[ \epsilon \right] =& \frac{\textrm{m}^2}{\textrm{s}^3}\\
\left[ \eta \right] =& \frac{1}{\textrm{s}^3}
\end{align} 

%%%%%%%%%%%%%%%%%%
%
\subsection{Important Scales}
%
%%%%%%%%%%%%%%%%%%

In terms of the forcing coefficient $f_\zeta$, energy is being injected at the rate $\epsilon=c_\epsilon f_\zeta^{\frac{3}{2}} k_f^{-2}$ and enstrophy is being injected at the rate $\eta= c_\epsilon (2\pi)^2 f_\zeta^{\frac{3}{2}}$. The factor of $2\pi$ appears because we are using a wavenumber in units of cycles per meter, rather than radians per meter. The derivative of the fourier component will therefore need the extra $2 \pi$ factor. The $c_\epsilon$ constant I'm adding because there does appear to be some constant needed to estimate this parameter, or more likely I'm missing something obvious. Either way, $c_\epsilon\approx 1/110$ provided that the forcing is correlated for longer than $f_\zeta^{-\frac{1}{2}}$, otherwise $c_\epsilon$ is smaller. I obtained this by computing $\epsilon$ and $\eta$ directly from the simulation and comparing with the above estimate. I'm pretty sure I should be able to figure out exactly what $c_\epsilon$ is, but for the moment it doesn't matter.

In a numerical simulation, the small scale dissipation is set to maintain numerical stability, and has almost no physical meaning. We will actually be choosing the scale at which we want dissipation to start to work (some multiple of the grid size), and then choose an appropriate coefficient based on the rate at which we're forcing. In the case of large scale energy, this has a (potentially) more physical meaning, but we again will choose a scale (a fraction of the domain width) and then set a coefficient large enough to remove the energy.

%%%%%%%%%%%%%%%%%%
\subsubsection{Large scale damping}
%%%%%%%%%%%%%%%%%%

In the case where $\alpha=0$, $r \neq 0$, the average energy must equilibrate to $\frac{\epsilon}{r}$ as can be seen from simply solving the differential equation. From scaling, we can determine the minimum scale at which friction matters is $k_r = \left( \frac{r^3}{\epsilon} \right)^{\frac{1}{2}}$. This implies a characteristic velocity scale of $u_r=\left(\frac{\epsilon}{k_r}\right)^\frac{1}{3}$. In practice, I seem to have to set $r = 0.04 \left( \epsilon k_r^2 \right)^{\frac{1}{3}}$ where I'm using $c_\epsilon=1$. The resulting $u_r$ works fantastically well for setting the time step, for a $cfl=0.25$.

In the case where $\alpha \neq 0$, $r = 0$, it's clear that the frictional scale must be $k_{\alpha}= \left( \frac{\alpha^3 \lambda^6}{\epsilon} \right)^{\frac{1}{8}}$. This implies a characteristic velocity scale of $u_\alpha=\left(\frac{\epsilon}{k_\alpha}\right)^\frac{1}{3}$. After a few tests it looks like this is a good estimate, but should be doubled. This could be a square root of energy (which needs an extra $\sqrt{2}$ factor).

Scott and Dritschel (2013) point out that when at, or beyond, the deformation scale $L_R$ we might expect the $k_\alpha$ to depend on $L_R$ as well. In their notation, $\lambda = L_R^{-1}$ and $k_\alpha$ is modified by a prefactor $C_h$,
\begin{equation}
k_{\alpha}= C_h \left( \frac{\epsilon k_f^2}{\alpha^3 \lambda^6} \right)^{\frac{1}{40}} \left( \frac{\alpha^3 \lambda^6}{\epsilon} \right)^{\frac{1}{8}}.
\end{equation}
This is their equation 3.7. When I reduce this, I find that,
\begin{equation}
k_{\alpha}= C_h \left( \frac{ k_f^2 \alpha^3 \lambda^6}{\epsilon} \right)^{\frac{1}{10}}.
\end{equation}
This isn't really all that different. The only difference is that it now depends on where it's being forced ($k_f$) rather than just the forcing intensity ($\epsilon$). This difference will only become apparent for flows above the deformation scale---presumably because the damping is a function of the stretching term, and now you're directly forcing the stretching. They find that $C_h\approx1.76$.

%%%%%%%%%%%%%%%%%%
\subsubsection{Small scale dissipation}
%%%%%%%%%%%%%%%%%%

Small scale dissipation will be chosen so that the Reynolds number is approximately $1$ at the grid scale. So we set $\nu_\alpha=u_\alpha \Delta x$ or $\nu_r=u_r \Delta x$ depending on which form of large scale damping we're using. In practice, we seem to be able to get away with viscosity a factor of 2 smaller. A factor of 4 maintains stability, but often results in some spectral blocking. Actually, in practice I find this not to be true. We do need to keep the Reynolds number at 1---there's no way around this.

From simple scaling arguments the enstrophy cascade will first be affect by damping when $ 2 \pi k_\nu=\left( \frac{ \eta}{\nu^3} \right)^{\frac{1}{6}}$. We introduced another factor of $2\pi$ because a derivative is taken to get from vorticity to vorticity damping. If our assumption about forcing is correct, this suggests that $\nu = f_\zeta^\frac{1}{2} (2 \pi k_\nu)^{-2}$.

The only complication is that we are using spectral vanishing viscosity to limit the effect of damping on lower wave numbers. In particular, if $k_{\textrm{max}}$ is the largest resolved wavenumber (taking into account anti-aliasing) the SVV operator prevents wave numbers below $k_{\textrm{cutoff}}=\Delta k \left(\frac{k_{\textrm{max}}}{\Delta k} \right)^{\frac{3}{4}}$ from being damped at all. Wavenumbers between the cutoff and the max have reduced damping by some factor $Q(k)$ where $0<Q(k)<1$. The wavenumber with $Q(k)$ reduction can be found with,
\begin{equation}
Q(k) = 
\begin{cases}
0 & k < k_{\textrm{cutoff}} \\
\exp \left[ - \left( \frac{k-k_{\textrm{max}}}{k-k_{\textrm{cutoff}}} \right)^2 \right] & k_{\textrm{cutoff}} < k < k_{\textrm{max}} \\
1 & k > k_{\textrm{max}}
\end{cases}
\end{equation}
The wavenumber at which viscosity is first felt, $k_\nu$ can be found by solving for $k_\nu$ in the equation $\nu Q(k_\nu) = f_\zeta^\frac{1}{2} (2 \pi k_\nu)^{-2}$

The CFL condition is $dt < \frac{(\Delta x)^2}{\nu}$.


\end{document}  