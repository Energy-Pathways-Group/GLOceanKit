\documentclass[11pt]{article}
\usepackage{geometry}                % See geometry.pdf to learn the layout options. There are lots.
\geometry{letterpaper}                   % ... or a4paper or a5paper or ... 
%\geometry{landscape}                % Activate for for rotated page geometry
%\usepackage[parfill]{parskip}    % Activate to begin paragraphs with an empty line rather than an indent
\usepackage{graphicx}
\usepackage{amssymb}
\usepackage{epstopdf}
\usepackage{amsmath}
\DeclareGraphicsRule{.tif}{png}{.png}{`convert #1 `dirname #1`/`basename #1 .tif`.png}

\title{Notes on Matlab 2D Boussinesq implementation}
\author{Jeffrey J. Early}
\date{March 12th, 2020}                                           % Activate to display a given date or no date

\begin{document}
\maketitle
%\section{}
%\subsection{}
The 2D Boussinesq equations are
\begin{subequations}{}
\begin{align}
\label{streamfunction_equation}
\nabla^2 \psi_t +  J\left( \psi, \nabla^2 \psi \right) =& \frac{g}{\rho_0}\rho_x + \nu Q(k) \psi_{xxxx} + \nu_z Q(m) \psi_{zzzz} \\ \label{rho_equation}
\rho_t + J\left( \psi, \rho \right) + \psi_x \bar{\rho}_z =& 0
\end{align}
\end{subequations}
where $J(a,b) \equiv a_x b_z - a_z b_x$. The linear solution written as a stream function is
\begin{align}
\label{linear_streamfunction}
\psi(x,z,t) =& U h \cos \theta G(z) \\
b(x,z,t) =& -U N^2 \frac{kh}{\omega} \cos \theta G(z)
\end{align}

Note we need to add damping on the density to match the damping that occurs on $w$.

We should write this in terms of buoyancy, $b(x,y,z,t) \equiv -\frac{g}{\rho_0} \rho$.
\begin{subequations}{}
\begin{align}
\rho_t + \psi_x \rho_z - \psi_z \rho_x + \psi_x \bar{\rho}_z =& 0 \\
b_t + \psi_x b_z - \psi_z b_x + \psi_x N^2 =& 0 \\
b_t =& \psi_z b_x -\psi_x (N^2 + b_z) 
\end{align}
\end{subequations}
So our equations to model are,

\begin{subequations}{}
\begin{align}
\nabla^2 \psi_t =& -  \psi_x \nabla^2 \psi_z + \psi_z \nabla^2 \psi_x  -b_x \\ 
b_t =& \psi_z b_x -\psi_x (N^2 + b_z) 
\end{align}
\end{subequations}

Using that,
\begin{align}
\psi(x,z) =& \psi_{km} e^{ikx} \sin mz \\
b(x,z) =& b_{km} e^{ikx} \sin mz \\
\end{align}
the derivatives of $\psi$ are,
\begin{align}
\psi_x =& ik\psi_{km} e^{ikx} \sin mz \\
\psi_z =& m \psi_{km} e^{ikx} \cos mz \\
\nabla^2 \psi =& -(k^2 + m^2) \psi_{km} e^{ikx} \sin mz \\
\nabla^2 \psi_x =& -i k (k^2 + m^2) \psi_{km} e^{ikx} \sin mz \\
\nabla^2 \psi_z =& -m (k^2 + m^2) \psi_{km} e^{ikx} \cos mz  \\
\psi_{xxxx} = & k^4 e^{ikx} \sin mz \\
\psi_{zzzz} = & m^4 e^{ikx} \sin mz
\end{align}
and the derivatives of $b$ are,
\begin{align}
b_x =& ik b_{km} e^{ikx} \sin mz \\
b_z =& m b_{km} e^{ikx} \cos mz
\end{align}

\end{document}  