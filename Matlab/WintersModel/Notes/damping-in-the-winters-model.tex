\documentclass[11pt, oneside]{article}   	% use "amsart" instead of "article" for AMSLaTeX format
\usepackage{geometry}                		% See geometry.pdf to learn the layout options. There are lots.
\geometry{letterpaper}                   		% ... or a4paper or a5paper or ... 
%\geometry{landscape}                		% Activate for rotated page geometry
%\usepackage[parfill]{parskip}    		% Activate to begin paragraphs with an empty line rather than an indent
\usepackage{graphicx}				% Use pdf, png, jpg, or eps§ with pdflatex; use eps in DVI mode
								% TeX will automatically convert eps --> pdf in pdflatex		
\usepackage{amssymb}
\usepackage{amsmath}

%SetFonts

%SetFonts


\title{Damping in the Winter's Model}
\author{Jeffrey J. Early}
%\date{}							% Activate to display a given date or no date

\begin{document}
\maketitle
%\section{}
%\subsection{}
Using the terminology of Kraig Winters we need to define a reasonable coefficient for damping.

Given that,
\begin{equation}
\frac{\partial u}{\partial t} = \nu \nabla^{2n} u
\end{equation}
is spectrally,
\begin{equation}
\frac{\partial u}{\partial t} = \nu (i k 2 \pi)^{2n} u
\end{equation}
which has the solution
\begin{equation}
u = e^{\nu (i k 2 \pi)^{2n} t}.
\end{equation}
We want to convert this into an e-fold time, so we want
\begin{equation}
e^{-t/T}=e^{\nu (i k 2 \pi)^{2n} t}.
\end{equation}
Using that $k=\frac{1}{2\Delta}$ where $\Delta$ is the sample interval and solving for $\nu$ in terms of the other variables,
\begin{equation}
\nu = \frac{(-1)^{n+1}}{T} \left( \frac{\Delta}{\pi} \right)^{2n}
\end{equation}

Now add some forcing,
\begin{align}
\frac{\partial \hat{u}}{\partial t} = \hat{F} +  \nu (i k 2 \pi)^{2n} \hat{u}
\end{align}
solution
\begin{equation}
\hat{u} = u_0 e^{\nu (-1)^n (k 2 \pi)^{2n} t} - (-1)^n \frac{F}{ \nu (k 2 \pi)^{2n}}
\end{equation}
Then, in steady state,
\begin{equation}
\hat{u}\hat{u}^\ast =  \frac{F^2}{ \nu^2 } (2 \pi k )^{-4n}
\end{equation}

\begin{itemize}
\item We can compute the wavenumber at which damping drops the amplitude more than 50 percent during the length of the simulation.
\item We can compute, given U, the cfl criteria, and then ask that the Reynolds number be one at the grid scale.
\end{itemize}

\end{document}  